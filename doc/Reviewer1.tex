\documentclass{article}
\usepackage[left=1in,right=1in,top=1in,bottom=1.3in]{geometry}
\usepackage{color}
\usepackage{hyperref}
\usepackage{natbib}

\title{
Manuscript: ``Characterization of the Deep-Water Surface Wave Variability 
in the California Current Region'' 
\\ (2017JC013280) \\ \vspace{.5cm}Response to the Reviewer's \#1 Comments}

\author{Ana. B. Villas B\^oas, Sarah. T. Gille, Matthew R. Mazloff, and  Bruce D. Cornuelle}

\begin{document}

\maketitle

{\color{blue}
We thank Reviewer \#1 for his/her review and suggestions. Below we address the comments from reviewer \#1 point by point.
}

\begin{enumerate}
\item
This is an interesting and valuable study, however, it needs some clarifications and justifications prior to publishing. Although marked for a major revision, it is somewhere in between the major and minor revisions.  
Some of the points for further clarifications and explanations include:

\begin{enumerate}
\item{
The authors correctly indicated expansion fans as areas with maximum winds generating upwelling and wind waves. However, since the maximum winds are fairly close to the coast, a question is how well the satellite detection can come close to the coast without contamination by the land. Also, CFSR data have coarse resolution and this should be discussed and emphasized what impact this coarse resolution has on final results.} 
\\\\
\indent \textcolor{blue}{ We agree with the reviewer that land contamination is an important issue and that current oceanographic satellites have heightened uncertainty in coastal zones. 
The goal of this manuscript, however, is to describe the \textbf{regional-scale}, \textbf{seasonal variability} of the \textbf{deep--water} surface wave field in the California Current region. Thus, the dynamics of processes happening within a few kilometers of the shoreline, the typical footprint size of satellite altimeters, are outside the scope of this study. We have emphasized our goal in the introduction of the revised manuscript (L. 97--98). Grid points at water depths shallower than 100 m or within less than 20 km from the coastline are not considered in our analysis (note the white spaces on the maps in Figures 2, 5, 6, 7, and 8). \textbf{This consideration is now explicitly stated in the methodology of the revised manuscript (L. 119--122, 167--169).}}
\\\\
\indent \textcolor{blue}{The same reasoning applies to the resolution of CFSR winds. It is outside the scope of the present work to resolve coastal dynamics/upwelling. Our discussion focuses on the impacts of regional-scale along-shore (northwesterly) winds, that are typical in April--July, in modulating the surface wave field. We would like to stress that scatterometer winds have been shown to capture fairly well the spatial structure of California--scale expansion fans, as well as high wind speeds off the major capes (e.g. Figure 1 of \cite{dorman2013impact} and Figure 9 of \cite{taylor2008northerly}). The same applies for wind speed retrieved from microwave radiometers (e.g. Figure 5 of \cite{koravcin2004coastal}). Finally, the fact that the same wind pattern, as observed by the authors cited above and the present study, projects onto all the wave parameters that we analyze (wave period, direction, significant wave height, and wave slope) is consistent with our assumption that the resolution of CFSR is adequate for the analysis that we have carried out.} 
\\
\item
L.117 - confirms problems with coarse resolution. A resolution of 1 x 1 deg is very coarse for investigating coastal dynamics. The authors can check how the coastline would be depicted with this type of resolution. Please explain.
\\\\
{\color{blue} We agree with the reviewer. Please refer to the previous items for a thorough discussion about resolution and the objectives of the present manuscript. Note that $1^\circ \times 1^\circ$ is the bin size used to average the along--track altimeter measurements of significant wave height. The only results computed at that resolution are the monthly maps of significant wave height and the timeseries at the SWOT calval site. Again, proper care was taken regarding proximity to land and the resolution of the data is adequate for the type of processes that we aim to describe.} \\\

\item L.158 - 0.5 deg resolution of WW3 further indicate a need to justify this resolution for the coastal region.
\\\\
{\color{blue} Please refer to the previous items for a thorough discussion about resolution and the objectives of the present manuscript. On this same note, it is important to keep in mind that our study relies on the combination of multiple data sets, capable of complementing each other. We have shown that the signature of expansion fan winds is captured by the wave model, the satellite altimeter, and by the wave buoys, which further supports the fact that the resolution of the data sets does not impact our results.}
\\
\item L. 131-134 - Since the CDIP buoys do not measure meteorology, the authors should add results from the NOAA weather buoys in the analysis.. See also L. 340
\\\\
{\color{blue} We thank the reviewer for raising this point. We agree that for obtaining a comprehensive description of the surface wind field variability and the dynamics of expansion fans, wind speed measurements from NOAA (NDBC) buoys would be an invaluable tool. However, the present manuscript focuses on surface gravity waves and the impact of remote versus local forcing in modulating the surface wave field. The dynamics of regional-scale winds off the California coast, more specifically expansion fans, are already well documented \cite[e.g][]{dorman2013impact, koravcin2004coastal, taylor2008northerly} and, therefore, outside the scope of our study.
\\\\
Since the focus of our study is on the waves, we have chosen to use CDIP buoys because CDIP uses Datawell Directional Waverider buoys, which are known to have more accurate directional spread and skewness in comparison to NOAA's 3-meter discus buoy \cite[e.g][]{o1996comparison, gallet2014refraction}. Part of this is associated with the fact that the CDIP buoys' only purpose is to measure waves, so their size and payload is optimized for that task, while NOAA buoys have meteorological instruments attached. As pointed out by reviewer 2, a valuable contribution of our manuscript is the analysis of the directional spectrum partitions, which requires reliable directional measurements of the wave field. \textbf{We have included the justification for choosing CDIP only in the revised manuscript (L. 128--131).}
\\\\
We would also like to emphasize that validation of CFSR surface winds against NDBC buoys is broadly available in the literature (L. 310--311 of the original manuscript). One example of that is Table 2 and Figure 8 of \cite{chawla2013validation}, and Figures 5.5 and 5.6 of \cite{splinder2011initial}. Comparisons of CFSR winds against \textbf{all} NDBC buoys in the West coast of the US is available in the supporting information of \cite{chawla2013validation}.
\\\\
Based on these, we have chosen not to add wind measurements from NDBC buoys to our analysis.}
\\\\
\item Another effect needs to be considered and discussed: During the warm season, periods of high winds and relaxation periods are interchanging and during the relaxation periods swell can be dominant.
\\\\
{\color{blue} This is correct. During spring and summer, in regions that are 
geographically exposed to south swell, both locally and remotely generated waves are observed. However, local winds deliver higher waves that, on average and in a climatological sense, contribute the most to the variability of the significant wave height at most sites (e.g. Figure 4). 
\textbf{We have addressed this topic in the discussion of Figure 4 (L. 243--252) and the discussion of Figure 9 (L. 368--370) of the revised manuscript}. 
}
\\\\
\item Please explain satellite vs. wind sampling periods.
\\\\
{\color{blue} We apologize, but we do not understand what the reviewer means by that. If this is
regarding the time span of the data sets, our analysis was performed only for the time that satellite altimetry, WW3 simulations, and CFSR winds overlap (1994--2012). If this is regarding the revisit time of the satellite altimeters, we use a multi-mission product \citep{queffeulou2004long}. The revisit period for individual satellites varies between 10 and 35 days. We do not believe this impacts our results since all the discussion is based on monthly averages taken over two decades.}
\\
\item L. 171-175 - explain why higher waves occur in winter.
\\\\
{\color{blue} The higher waves in the winter are a consequence of winter cyclone/anticyclone 
systems that propagate from the northwestern Pacific into the Gulf of Alaska. This is explained in L. 182--184 of the revised manuscript.}
\\
\item L. 182 - 186 - can you explain why. \\\\
{\color{blue} The localized region of $H_s \geq 2$ m in April-July reflects the effect of regional-scale
expansion fan winds on the wave field. This point is discussed throughout the Results section and 
summarized in L 502--506 of the revised manuscript.}\\

\item L. 195 -198 - this should be supported by sound evidence or removed.\\\\
{\color{blue} Removed.}
\\
\item L. 221 - summer waves - is this related to swell?
\\\\
{\color{blue} Yes. This is related to swell from storms in the Southern Ocean. However, we do not discuss this at that point of the paper since it would be inappropriate to conclude anything about swell versus local waves before analyzing the wave period. We make this point clear latter on L. 274--276, as well as in the discussion of the spectrum partitions (L. 414--420).  
}
\\\\
\item L. 246 - Fig. 5 - How the vector fields (May-July) relate to upwelling winds?
\\\\
{\color{blue} Thank you for this comment. We have updated Figure 5, plotting the vectors more frequently. We hope this will give a better representation of the vector field. Expansion fan winds generate waves that have peak direction from the WNW to N. This becomes evident when we perform the analysis of the spectrum partitions. Note that the monthly average peak directions of Figure 5 should not exactly match the wind direction since both local and remote waves are considered in the averaging.}
\\
\item L.316 - 318 - explain more a synoptic setup, i.e., synoptic setup over the sw U.S.
\\\\

{\color{blue} 
During boreal winter the wind field in the eastern Pacific is mostly influenced by two major pressure systems: the Aleutian low (centered at about 50$^\circ$N, between the dateline and $170^\circ$W) and the North Pacific high (centered around $30^\circ$N and 135$^\circ$W). These pressure systems drive stronger (7--8 m/s) southwesterly winds off the California coast north of 40$^\circ$N and weaker (4--6 m/s) northwesterly winds south of 40$^\circ$N.
In spring/summer, the northward migration of the North Pacific high together with the development of a thermal low over the southwestern US shifts the mean wind towards a more northwesterly orientation along the entire California coast.
These northwesterly winds lead to a low-level inversion that caps the marine atmospheric boundary layer at heights lower than the coastal topography, such that the atmospheric flow is channeled. As these alongshore winds approach a cape, regions of compression (deceleration) are expected to develop upwind of the cape, followed by regions of expansion (acceleration) downwind of it.
\\\\
\textbf{Please see L. 52--68 and L. 293--311 of the revised manuscript}.}


\end{enumerate}



\end{enumerate}

\bibliographystyle{chicago}
\bibliography{Reviewer1}

\end{document}
